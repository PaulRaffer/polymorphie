\newpage
\section{Polymorphie}
	Mithilfe von Polymophie kann man das gleiche Interface für Objekte von verschieden Typen bereitstellen. Polymorphie wird
	auch Polymorphismus oder Vielgestaltigkeit genannt. Das Gegenteil von Polymorphie ist Monomothie. Die meisten Codebeispiele
	in diesem Dokument sind in C++ geschrieben, da C++ alle der hier genannten Formen der Polymorphie unterstützt.
	
	\subsection{Geschischte}
	
	\subsection{Arten der Polymophie}
		\begin{table}[h]
	\begin{tabularx}{\textwidth}{$l|^X|^X}
		\rowstyle{\bfseries}  & universell                              & Ad-hoc                             \\
		\rowstyle{\small}     & unendlich viele Typen                   & endliche Anzahl an Typen           \\
		\rowstyle{\small}     & eine Implementierung                    & unterschiedliche Implementierungen \\
		\hline
		\bfseries dynamisch   & \multirow{2}{*}{Inklusionspolymorphie/} &                                    \\
		\small    Laufzeit    & \multirow{2}{*}{Vererbungspolymorphie}  &                                    \\
		\small    langsamer   &                                         &                                    \\
		\hline
		\bfseries statisch    & \multirow{2}{*}{parametrische}          & \multirow{2}{*}{Überladung,}        \\
		\small    Kompilezeit & \multirow{2}{*}{Polymorphie}            & \multirow{2}{*}{Coercion}          \\
		\small    schneller   &                                         &                                    \\
	\end{tabularx}
\end{table}


		\subsubsection{universelle Polymorphie}
			Mithilfe universeller Polymorphie kann das gleiche Interface für unendlich viele Typen (auch Solche, die es noch
			gar nicht gibt, sondern die erst in der Zukunft Programmiert werden) bereitgestellt werden, darum wird sie auch
			``echte'' Vielgestaltigkeit genannt.
			
			\paragraph{Inklusionspolymorphie}
				
			
			\paragraph{Vererbungspolymorphie}
				In objektorientierten Programmiersprachen wird Inklusionspolymorphie meist durch Vererbung ausgedrückt.
				Trotzdem sind Inklusionspolymorphie und Vererbungspolymorphie nicht das Selbe, da sich Vererbungspolymorphie
				nicht an die strengen Regeln der Inklusionspolymorphie halten muss.
				Viele
				Klassen sind miteinander verwandt. Diese Beziehungen werden durch Vererbung dargestellt. Vererbungspolymorphie
				ist der Inklusionspolymorphie ähnlich, jedoch nicht das Selbe!
				
				Wenn z.B. Klasse B von Klasse A aberbt, werden alle Eigenschaften und Methoden der Basisklasse A (auch Ober-,
				Super- oder Elternklasse) in die abgeleitete Klass B (auch Unter-, Sub- oder Kindklasse) übernommen.
				
				Vererbung stellt meistens eine ist-ein-Bezierhung da. 
				Jedoch sollte nicht jede ist-ein-Beziehung durch Vererbung dargestellt werden. (siehe Kreis-Ellipse-Problem)
				
				\subparagraph{Schnittstellenvererbung}
				
				\subparagraph{Implementierungsvererbung}
					Bei der Implementierungsvererbung wird im Gegensatz zur Schnittstellenverebung nicht nur das Interface
					sondern auch die Implementierung der Methoden in die abgeleitete Klasse übernommen. So kann
					beispielsweise von der Basisklasse ein Standardverhalten vorgeschlagen werden, welches aber wenn dies
					Notwendig ist, von der abgeleiteten Klasse überschrieben werden kann.
				
				\subparagraph{Datenkapselung}
				sdfsa
					\begin{table}[h]
	\begin{tabularx}{\textwidth}{$X|^l|^X|^X|^X}
		\rowstyle{\bfseries} Zugriffs\-modifikator & UML & Eigene Klasse & Abgeleitete Klasse & Außerhalb      \\
		\hline
		\bfseries            privat                & -   & sichtbar      & nicht sichtbar     & nicht sichtbar \\
		\bfseries            geschützt             & \#  & sichtbar      & sichtbar           & nicht sichtbar \\
		\bfseries            öffentlich            & +   & sichtbar      & sichtbar           & sichtbar       \\
	\end{tabularx}
\end{table}
					
					\begin{table}[h]
	\begin{tabularx}{\textwidth}{$l|^X|^X|^X}
		                     \multicolumn{4}{l}{\bfseries Sichtbarkeit in ...}                        \\
		                     \bfseries Basisklasse    & \multicolumn{3}{l}{\bfseries Unterklasse}     \\
		\rowstyle{\bfseries} Art der Vererbung:       & öffentlich    & privat        & geschützt     \\
		\hline
		                     privat     $\rightarrow$ & nicht vererbt & nicht vererbt & nicht vererbt \\
		                     geschützt  $\rightarrow$ & geschützt     & privat        & geschützt     \\
		                     öffentlich $\rightarrow$ & öffentlich    & privat        & geschützt     \\
	\end{tabularx}
\end{table}

					
					In der Praxis wird meistens öffentliche Vererbung (ist ein) verwendet.
				
				\subparagraph{Mehrfachvererbung}
					Wenn eine Unterklasse von mehr als einer Oberklasse aberbt spricht man von Mehrfachvererbung.
					Mehrfachinterfacevererbung ist in den meisten fällenproblemlos möglich. Mehrfachimplementierungsvererbung
					jedoch lassen viele Sprachen nicht zu, da diese oft zu fehleranfälligem und unübersichtlichem Code führt,
					da sich Implementierungen, welche aus unterschiedlichen Basisklassen vererbt werden, wiedersprechen
					könnten.
					
					\subparagraph{Virtuelle Vererbung}
						Eines der Probleme zu denen Mehrfachvererbung führen kann ist das Diamond-Problem. Dieses tritt auf,
						wenn eine abgeleitete Klasse über mehr als einen Pfad von derselben Basisklasse aberbt. In C++ kann
						das Problem mithilfe von virtueller Vererbung gelöst werden. Hierbei erben alle Klassen, die direkt
						von der in der Vererbungshierachie höchsten Oberklasse aberben, virtuell von dieser ab und teilen sich
						somit eine gemeinsame Instanz von dieser Klasse.
				
				\subparagraph{Abstrakte Klassen}
					Abstrakte Klassen sind Klassen, von der keine Objekte erzeugt werden können. Das macht natürlich nur Sinn,
					wenn von so einer Klasse abgeerbt wird und von einer Unterklasse ein Objekt erzeugt wird.
				
				\subparagraph{Entgültige Klassen}
					Entgültige Klasssen sind Klassen von denen nicht abgeerbt werden kann.
					
				\subparagraph{Liskovsches Substitutionsprinzip}
					Das Liskovsche Substitutionsprinzip ist erfüllt, wenn jedes Objekt einer Oberklasse problemlos durch ein
					Objekt dessen Unterklasse ersetzt werden kann.
					
					\subparagraph{Kreis-Ellipse-Problem}
						Das Kreis-Ellipse-Problem (bzw. Quadrat-Rechteck-Problem) ist ein Problem in der objektorientierten
						Programmierung.
						
						Lösungsvorschläge:
						
						\begin{itemize}
							\item Fehler bei Größenänderung:
								Nachteil: Problem muss auf Ebene der Basisklasse gelöst werden
							\item Ellipse erbt von Kreis ab:
							
							\item Keine Klasse Kreis
							
							\item Keine Vererbungsbeziehung zwischen Ellipse und Kreis
								Nachteil: Doppelte Implementierung
							\item Einführen neuer Basisklasse
							
						\end{itemize}
			
				
			\paragraph{Parametrische Polymorphie}
				\subparagraph{einfachen parametrischen Polymorphismus}
				
				\subparagraph{beschränkten parametrischen Polymorphismus}
					Im Gegensatz zum einfachen parametrischen Polymorphismus ist die beschränkte parametrischen Polymorphismus
					typensicher.
					
			
		\subsubsection{Ad-hoc-Polymorphie}
			Mithilfe von Ad-hoc-Polymorphie kann das gleiche Interface im Gegensatz zu universellen Polymorphie nur für eine
			begrengte Anzahl an bestimmten Typen bereitgestellt werden.
		
			\paragraph{Coercion}
				Coercion ist eine implizite Typumwandlung vom Compiler.
			
			\paragraph{Überladung}
				\subparagraph{Operatorüberladung}
					Manche Programmiersprachen (darunter C++) ermöglichen das Überladen von Operatoren für eigene Typen. Die
					Operatorüberladung ist ein Spezialfall der Funktionsüberladung, da Operatoren auch nichts anderes als
					Funktionen mit einem oder zwei Parametern sind.
					
					Es können beispielsweise die Arithmetischen Operatoren für eine eigene Bruchklasse definiert werden.
					
\newpage
\section{Quellen}
	\begin{itemize}
	\item FSST-Mitschrift der 2. und 3. Klasse
	\item \url{https://www.ics.uci.edu/~jajones/INF102-S18/readings/05_stratchey_1967.pdf}
	\item \url{https://www.youtube.com/watch?v=uTxRF5ag27A&list=PLrAXtmErZgOdP_8GztsuKi9nrraNbKKp4}
	\item \url{https://soundcloud.com/lambda-cast}
	\item \url{https://de.wikipedia.org/wiki/Polymorphie_(Programmierung)}
	\item \url{https://lec.inf.ethz.ch/ifmp/2019/slides/lecture14.handout.pdf}
	\item \url{https://courses.cs.washington.edu/courses/cse331/11wi/lectures/lect06-subtyping.pdf}
	\item \url{https://de.wikipedia.org/wiki/Vererbung_(Programmierung)}
	\item \url{https://wiki.haskell.org/Polymorphism}
	\item \url{http://lucacardelli.name/Papers/OnUnderstanding.A4.pdf}
	\item C++ Templates: The Complete Guide
	\item C++: Das umfassende Handbuch
	\item Grundkurs C++
	\item C++: Die Sprache der Objekte
	\item Exceptional C++: 47 engineering puzzles, programming problems, and solutions
	\item Effektiv C++. 50 Specific Ways to Improve Your Programs and Designs
	\item \url{https://www.youtube.com/watch?v=HddFGPTAmtU}
	\item \url{https://www.bfilipek.com/2020/04/variant-virtual-polymorphism.html}
	\item \url{https://de.wikipedia.org/wiki/Diamond-Problem}
\end{itemize}

	
